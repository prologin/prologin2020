% This file was generated by stechec2-generator. DO NOT EDIT.

\noindent \begin{tabular}{lp{11cm}}
\textbf{Constante:} & NB\_JARDINIERS \\
\textbf{Valeur:} & 2 \\
\textbf{Description:} & Nombre de jardiniers \\
\end{tabular}
\vspace{0.2cm} \\

\noindent \begin{tabular}{lp{11cm}}
\textbf{Constante:} & TAILLE\_GRILLE \\
\textbf{Valeur:} & 20 \\
\textbf{Description:} & Largeur et hauteur de la grille \\
\end{tabular}
\vspace{0.2cm} \\

\noindent \begin{tabular}{lp{11cm}}
\textbf{Constante:} & NB\_TOURS \\
\textbf{Valeur:} & 100 \\
\textbf{Description:} & Nombre de tours à jouer avant la fin de la partie \\
\end{tabular}
\vspace{0.2cm} \\

\noindent \begin{tabular}{lp{11cm}}
\textbf{Constante:} & AGE\_MAX \\
\textbf{Valeur:} & 10 \\
\textbf{Description:} & Durée de vie maximale d'une plante \\
\end{tabular}
\vspace{0.2cm} \\

\noindent \begin{tabular}{lp{11cm}}
\textbf{Constante:} & AGE\_DE\_POUSSE \\
\textbf{Valeur:} & 3 \\
\textbf{Description:} & Âge auquel la plante atteint la maturité et peut donc être arrosée \\
\end{tabular}
\vspace{0.2cm} \\

\noindent \begin{tabular}{lp{11cm}}
\textbf{Constante:} & PORTEE\_BAFFE \\
\textbf{Valeur:} & 6 \\
\textbf{Description:} & Portée maximale d'une baffe \\
\end{tabular}
\vspace{0.2cm} \\

\noindent \begin{tabular}{lp{11cm}}
\textbf{Constante:} & NB\_TYPES\_RESSOURCES \\
\textbf{Valeur:} & 3 \\
\textbf{Description:} & Types de ressources existantes \\
\end{tabular}
\vspace{0.2cm} \\

\noindent \begin{tabular}{lp{11cm}}
\textbf{Constante:} & APPORT\_CARACTERISTIQUE \\
\textbf{Valeur:} & 10 \\
\textbf{Description:} & Apport pour une caractéristique lors de l'arrosage \\
\end{tabular}
\vspace{0.2cm} \\

\noindent \begin{tabular}{lp{11cm}}
\textbf{Constante:} & COUT\_PAR\_CASE\_COLLECTE \\
\textbf{Valeur:} & 15 \\
\textbf{Description:} & Nombre de points de caractéristiques nécessaires pour pouvoir dépoter une cases plus loins. \\
\end{tabular}
\vspace{0.2cm} \\

\noindent \begin{tabular}{lp{11cm}}
\textbf{Constante:} & COUT\_PAR\_CASE\_DEPOTAGE \\
\textbf{Valeur:} & 15 \\
\textbf{Description:} & Nombre de points de caractéristiques nécessaires pour pouvoir dépoter une case plus loin. \\
\end{tabular}
\vspace{0.2cm} \\


\functitle{erreur} \\
\noindent
\begin{tabular}[t]{@{\extracolsep{0pt}}>{\bfseries}lp{10cm}}
Description~: & Erreurs possibles \\
Valeurs~: &
\small
\begin{tabular}[t]{@{\extracolsep{0pt}}lp{7cm}}
    \textsl{OK}~: & L'action s'est effectuée avec succès \\
    \textsl{HORS\_TOUR}~: & Il est interdit de faire des actions hors de jouer\_tour \\
    \textsl{HORS\_POTAGER}~: & La case désignée n'est pas dans le potager \\
    \textsl{CASE\_OCCUPEE}~: & Il y a déjà une plante sur la case ciblée \\
    \textsl{PAS\_DE\_PLANTE}~: & Il n'y a pas de plante sur la case ciblée \\
    \textsl{MAUVAIS\_JARDINIER}~: & La plante n'appartient pas au bon jardinier \\
    \textsl{SANS\_POT}~: & La plante est déjà dépotée \\
    \textsl{DEJA\_ARROSEE}~: & La plante a déjà été arrosée \\
    \textsl{DEJA\_BAFFEE}~: & La plante a déjà baffé ce tour ci \\
    \textsl{PAS\_ENCORE\_ARROSEE}~: & La plante n'a pas encore été arrosée \\
    \textsl{PAS\_ENCORE\_ADULTE}~: & La plante ne peut pas encore être arrosée \\
    \textsl{PLANTE\_INVALIDE}~: & Les caractéristiques de la plante sont invalides \\
    \textsl{TROP\_LOIN}~: & La plante n'a pas un assez grand rayon de dépotage \\
    \textsl{CARACTERISTIQUE\_INVALIDE}~: & Valeur de `Caracteristique` inconnue \\
    \textsl{CHIEN\_INVALIDE}~: & Valeur de `Chien` inconnue \\
\end{tabular} \\
\end{tabular}

\functitle{action\_type} \\
\noindent
\begin{tabular}[t]{@{\extracolsep{0pt}}>{\bfseries}lp{10cm}}
Description~: & Types d'actions \\
Valeurs~: &
\small
\begin{tabular}[t]{@{\extracolsep{0pt}}lp{7cm}}
    \textsl{ACTION\_DEPOTER}~: & Action ``depoter`` \\
    \textsl{ACTION\_BAFFER}~: & Action ``baffer`` \\
    \textsl{ACTION\_ARROSER}~: & Action ``arroser`` \\
\end{tabular} \\
\end{tabular}

\functitle{caracteristique} \\
\noindent
\begin{tabular}[t]{@{\extracolsep{0pt}}>{\bfseries}lp{10cm}}
Description~: & Caractéristiques améliorables d'une plante \\
Valeurs~: &
\small
\begin{tabular}[t]{@{\extracolsep{0pt}}lp{7cm}}
    \textsl{CARACTERISTIQUE\_FORCE}~: & Force \\
    \textsl{CARACTERISTIQUE\_VIE}~: & Vie  \\
    \textsl{CARACTERISTIQUE\_ELEGANCE}~: & Élégance \\
    \textsl{CARACTERISTIQUE\_RAYON\_DEPOTAGE}~: & Portée de dépotage \\
\end{tabular} \\
\end{tabular}

\functitle{debug\_chien} \\
\noindent
\begin{tabular}[t]{@{\extracolsep{0pt}}>{\bfseries}lp{10cm}}
Description~: & Types de chien de débug \\
Valeurs~: &
\small
\begin{tabular}[t]{@{\extracolsep{0pt}}lp{7cm}}
    \textsl{AUCUN\_CHIEN}~: & Aucun chien, enlève le chien présent \\
    \textsl{CHIEN\_BLEU}~: & Chien bleu \\
    \textsl{CHIEN\_VERT}~: & Chien vert \\
    \textsl{CHIEN\_ROUGE}~: & Chien rouge \\
\end{tabular} \\
\end{tabular}



\functitle{position}
\begin{lst-c++}
struct position {
    int x;
    int y;
};
\end{lst-c++}
\noindent
\begin{tabular}[t]{@{\extracolsep{0pt}}>{\bfseries}lp{10cm}}
Description~: & Position dans le jardin, donnée par deux coordonnées. \\
Champs~: &
\small
\begin{tabular}[t]{@{\extracolsep{0pt}}lp{7cm}}
    \textsl{x}~: & Coordonnée : x \\
    \textsl{y}~: & Coordonnée : y \\
\end{tabular} \\
\end{tabular}

\functitle{plante}
\begin{lst-c++}
struct plante {
    position pos;
    int jardinier;
    bool adulte;
    bool enracinee;
    int vie;
    int vie_max;
    int force;
    int elegance;
    int rayon_deplacement;
    int rayon_collecte;
    int array consommation;
    int age;
};
\end{lst-c++}
\noindent
\begin{tabular}[t]{@{\extracolsep{0pt}}>{\bfseries}lp{10cm}}
Description~: & Une plante \\
Champs~: &
\small
\begin{tabular}[t]{@{\extracolsep{0pt}}lp{7cm}}
    \textsl{pos}~: & Position de la plante \\
    \textsl{jardinier}~: & Jardinier ayant planté la plante \\
    \textsl{adulte}~: & La plante est adulte \\
    \textsl{enracinee}~: & La plante a déjà été dépotée \\
    \textsl{vie}~: & Point(s) de vie restant(s) de la plante \\
    \textsl{vie\_max}~: & Point(s) de vie maximum de la plante \\
    \textsl{force}~: & Force de la baffe de la plante \\
    \textsl{elegance}~: & Élégance de la plante \\
    \textsl{rayon\_deplacement}~: & Distance maximale parcourable par la plante en creusant \\
    \textsl{rayon\_collecte}~: & Rayon de collecte des ressources pour la plante \\
    \textsl{consommation}~: & Quantité de ressources consommées par la plante \\
    \textsl{age}~: & Âge de la plante \\
\end{tabular} \\
\end{tabular}

\functitle{action\_hist}
\begin{lst-c++}
struct action\_hist {
    action_type atype;
    position position_baffante;
    position position_baffee;
    position position_depart;
    position position_arrivee;
    position position_plante;
    caracteristique amelioration;
};
\end{lst-c++}
\noindent
\begin{tabular}[t]{@{\extracolsep{0pt}}>{\bfseries}lp{10cm}}
Description~: & Représentation d'une action dans l'historique \\
Champs~: &
\small
\begin{tabular}[t]{@{\extracolsep{0pt}}lp{7cm}}
    \textsl{atype}~: & Type de l'action \\
    \textsl{position\_baffante}~: & Position de la plante baffante (si type d'action ``action\_baffer``) \\
    \textsl{position\_baffee}~: & Position de la plante baffée (si type d'action ``action\_baffer``) \\
    \textsl{position\_depart}~: & Position de la plante à déplacer (si type d'action ``action\_depoter``) \\
    \textsl{position\_arrivee}~: & Position où déplacer la plante (si type d'action ``action\_depoter``) \\
    \textsl{position\_plante}~: & Position de la plante  (si type d'action ``action\_arroser``) \\
    \textsl{amelioration}~: & Caractéristique à améliorer (si type d'action ``action\_arroser``) \\
\end{tabular} \\
\end{tabular}



\begin{minipage}{\linewidth}
\functitle{depoter}
\begin{lst-c++}
erreur depoter(position position_depart, position position_arrivee)
\end{lst-c++}
\noindent
\begin{tabular}[t]{@{\extracolsep{0pt}}>{\bfseries}lp{10cm}}
Description~: & La plante creuse vers une destination donnée \\
Paramètres~: &
\begin{tabular}[t]{@{\extracolsep{0pt}}ll}
    \textsl{position\_depart}~: & Position de la plante à déplacer \\
    \textsl{position\_arrivee}~: & Position où déplacer la plante \\
  \end{tabular} \\
\end{tabular} \\[0.3cm]
\end{minipage}

\begin{minipage}{\linewidth}
\functitle{arroser}
\begin{lst-c++}
erreur arroser(position position_plante, caracteristique amelioration)
\end{lst-c++}
\noindent
\begin{tabular}[t]{@{\extracolsep{0pt}}>{\bfseries}lp{10cm}}
Description~: & Arrose une plante \\
Paramètres~: &
\begin{tabular}[t]{@{\extracolsep{0pt}}ll}
    \textsl{position\_plante}~: & Position de la plante \\
    \textsl{amelioration}~: & Caractéristique à améliorer \\
  \end{tabular} \\
\end{tabular} \\[0.3cm]
\end{minipage}

\begin{minipage}{\linewidth}
\functitle{baffer}
\begin{lst-c++}
erreur baffer(position position_baffante, position position_baffee)
\end{lst-c++}
\noindent
\begin{tabular}[t]{@{\extracolsep{0pt}}>{\bfseries}lp{10cm}}
Description~: & Une plante en gifle une autre \\
Paramètres~: &
\begin{tabular}[t]{@{\extracolsep{0pt}}ll}
    \textsl{position\_baffante}~: & Position de la plante baffante \\
    \textsl{position\_baffee}~: & Position de la plante baffée \\
  \end{tabular} \\
\end{tabular} \\[0.3cm]
\end{minipage}

\begin{minipage}{\linewidth}
\functitle{debug\_afficher\_chien}
\begin{lst-c++}
erreur debug_afficher_chien(position pos, debug_chien chien)
\end{lst-c++}
\noindent
\begin{tabular}[t]{@{\extracolsep{0pt}}>{\bfseries}lp{10cm}}
Description~: & Affiche le chien spécifié sur la case indiquée \\
Paramètres~: &
\begin{tabular}[t]{@{\extracolsep{0pt}}ll}
    \textsl{pos}~: & Case ciblée \\
    \textsl{chien}~: & Chien à afficher sur la case \\
  \end{tabular} \\
\end{tabular} \\[0.3cm]
\end{minipage}

\begin{minipage}{\linewidth}
\functitle{plantes\_jardinier}
\begin{lst-c++}
plante array plantes_jardinier(int jardinier)
\end{lst-c++}
\noindent
\begin{tabular}[t]{@{\extracolsep{0pt}}>{\bfseries}lp{10cm}}
Description~: & Renvoie la liste des plantes du jardinier \\
Paramètres~: &
\begin{tabular}[t]{@{\extracolsep{0pt}}ll}
    \textsl{jardinier}~: & ID du jardinier concerné \\
  \end{tabular} \\
\end{tabular} \\[0.3cm]
\end{minipage}

\begin{minipage}{\linewidth}
\functitle{plante\_sur\_case}
\begin{lst-c++}
plante plante_sur_case(position pos)
\end{lst-c++}
\noindent
\begin{tabular}[t]{@{\extracolsep{0pt}}>{\bfseries}lp{10cm}}
Description~: & Renvoie la plante sur la position donnée, s'il n'y en a pas tous les champs sont initialisés à -1 \\
Paramètres~: &
\begin{tabular}[t]{@{\extracolsep{0pt}}ll}
    \textsl{pos}~: & Case ciblée \\
  \end{tabular} \\
\end{tabular} \\[0.3cm]
\end{minipage}

\begin{minipage}{\linewidth}
\functitle{plantes\_arrosables}
\begin{lst-c++}
plante array plantes_arrosables(int jardinier)
\end{lst-c++}
\noindent
\begin{tabular}[t]{@{\extracolsep{0pt}}>{\bfseries}lp{10cm}}
Description~: & Renvoie la liste des plantes du jardinier qui peuvent être arrosées \\
Paramètres~: &
\begin{tabular}[t]{@{\extracolsep{0pt}}ll}
    \textsl{jardinier}~: & ID du jardinier concerné \\
  \end{tabular} \\
\end{tabular} \\[0.3cm]
\end{minipage}

\begin{minipage}{\linewidth}
\functitle{plantes\_adultes}
\begin{lst-c++}
plante array plantes_adultes(int jardinier)
\end{lst-c++}
\noindent
\begin{tabular}[t]{@{\extracolsep{0pt}}>{\bfseries}lp{10cm}}
Description~: & Renvoie la liste des plantes du jardinier qui sont adultes \\
Paramètres~: &
\begin{tabular}[t]{@{\extracolsep{0pt}}ll}
    \textsl{jardinier}~: & ID du jardinier concerné \\
  \end{tabular} \\
\end{tabular} \\[0.3cm]
\end{minipage}

\begin{minipage}{\linewidth}
\functitle{plantes\_depotables}
\begin{lst-c++}
plante array plantes_depotables(int jardinier)
\end{lst-c++}
\noindent
\begin{tabular}[t]{@{\extracolsep{0pt}}>{\bfseries}lp{10cm}}
Description~: & Renvoie la liste des plantes du jardinier qui peuvent être dépotées \\
Paramètres~: &
\begin{tabular}[t]{@{\extracolsep{0pt}}ll}
    \textsl{jardinier}~: & ID du jardinier concerné \\
  \end{tabular} \\
\end{tabular} \\[0.3cm]
\end{minipage}

\begin{minipage}{\linewidth}
\functitle{ressources\_sur\_case}
\begin{lst-c++}
int array ressources_sur_case(position pos)
\end{lst-c++}
\noindent
\begin{tabular}[t]{@{\extracolsep{0pt}}>{\bfseries}lp{10cm}}
Description~: & Renvoie les ressources disponibles sur une case donnée \\
Paramètres~: &
\begin{tabular}[t]{@{\extracolsep{0pt}}ll}
    \textsl{pos}~: & Case ciblée \\
  \end{tabular} \\
\end{tabular} \\[0.3cm]
\end{minipage}

\begin{minipage}{\linewidth}
\functitle{reproduction\_possible}
\begin{lst-c++}
bool reproduction_possible(position pos, int rayon_collecte, int array consommation)
\end{lst-c++}
\noindent
\begin{tabular}[t]{@{\extracolsep{0pt}}>{\bfseries}lp{10cm}}
Description~: & Vérifie si une plante à la position donnée aura suffisamment de ressources pour se reproduire. S'il y a déjà une plante à cette position, le calcul suposera qu'elle a été remplacée \\
Paramètres~: &
\begin{tabular}[t]{@{\extracolsep{0pt}}ll}
    \textsl{pos}~: & Case ciblée \\
    \textsl{rayon\_collecte}~: & Rayon de collecte des ressources pour la plante \\
    \textsl{consommation}~: & Quantité de ressources consommées par la plante \\
  \end{tabular} \\
\end{tabular} \\[0.3cm]
\end{minipage}

\begin{minipage}{\linewidth}
\functitle{plante\_reproductible}
\begin{lst-c++}
bool plante_reproductible(position pos)
\end{lst-c++}
\noindent
\begin{tabular}[t]{@{\extracolsep{0pt}}>{\bfseries}lp{10cm}}
Description~: & Vérifie si une plante à la position donnée peut se reproduire, retourne faux s'il n'y pas de plante à la position donnée \\
Paramètres~: &
\begin{tabular}[t]{@{\extracolsep{0pt}}ll}
    \textsl{pos}~: & Case ciblée \\
  \end{tabular} \\
\end{tabular} \\[0.3cm]
\end{minipage}

\begin{minipage}{\linewidth}
\functitle{croisement}
\begin{lst-c++}
plante croisement(plante array parents)
\end{lst-c++}
\noindent
\begin{tabular}[t]{@{\extracolsep{0pt}}>{\bfseries}lp{10cm}}
Description~: & Caractéristiques d'une plante résultant du croisement de plusieurs parents donnés. Les champs sont initialisés à -1 si aucune plante n'est donnée en paramètre \\
Paramètres~: &
\begin{tabular}[t]{@{\extracolsep{0pt}}ll}
    \textsl{parents}~: & Les plantes qui sont croisées \\
  \end{tabular} \\
\end{tabular} \\[0.3cm]
\end{minipage}

\begin{minipage}{\linewidth}
\functitle{historique}
\begin{lst-c++}
action_hist array historique()
\end{lst-c++}
\noindent
\begin{tabular}[t]{@{\extracolsep{0pt}}>{\bfseries}lp{10cm}}
Description~: & Renvoie la liste des actions effectuées par l’adversaire durant son tour, dans l'ordre chronologique. Les actions de débug n'apparaissent pas dans cette liste. \\
\end{tabular} \\[0.3cm]
\end{minipage}

\begin{minipage}{\linewidth}
\functitle{score}
\begin{lst-c++}
int score(int id_jardinier)
\end{lst-c++}
\noindent
\begin{tabular}[t]{@{\extracolsep{0pt}}>{\bfseries}lp{10cm}}
Description~: & Renvoie le score du jardinier ``id\_jardinier``. Renvoie -1 si le jardinier est invalide. \\
Paramètres~: &
\begin{tabular}[t]{@{\extracolsep{0pt}}ll}
    \textsl{id\_jardinier}~: & Numéro du jardinier \\
  \end{tabular} \\
\end{tabular} \\[0.3cm]
\end{minipage}

\begin{minipage}{\linewidth}
\functitle{moi}
\begin{lst-c++}
int moi()
\end{lst-c++}
\noindent
\begin{tabular}[t]{@{\extracolsep{0pt}}>{\bfseries}lp{10cm}}
Description~: & Renvoie votre numéro de jardinier. \\
\end{tabular} \\[0.3cm]
\end{minipage}

\begin{minipage}{\linewidth}
\functitle{adversaire}
\begin{lst-c++}
int adversaire()
\end{lst-c++}
\noindent
\begin{tabular}[t]{@{\extracolsep{0pt}}>{\bfseries}lp{10cm}}
Description~: & Renvoie le numéro du jardinier adverse. \\
\end{tabular} \\[0.3cm]
\end{minipage}

\begin{minipage}{\linewidth}
\functitle{annuler}
\begin{lst-c++}
bool annuler()
\end{lst-c++}
\noindent
\begin{tabular}[t]{@{\extracolsep{0pt}}>{\bfseries}lp{10cm}}
Description~: & Annule la dernière action. Renvoie faux quand il n'y a pas d'action à annuler ce tour ci. \\
\end{tabular} \\[0.3cm]
\end{minipage}

\begin{minipage}{\linewidth}
\functitle{tour\_actuel}
\begin{lst-c++}
int tour_actuel()
\end{lst-c++}
\noindent
\begin{tabular}[t]{@{\extracolsep{0pt}}>{\bfseries}lp{10cm}}
Description~: & Retourne le numéro du tour actuel. \\
\end{tabular} \\[0.3cm]
\end{minipage}

\begin{minipage}{\linewidth}
\functitle{afficher\_erreur}
\begin{lst-c++}
void afficher_erreur(erreur v)
\end{lst-c++}
\noindent
\begin{tabular}[t]{@{\extracolsep{0pt}}>{\bfseries}lp{10cm}}
Description~: & Affiche le contenu d'une valeur de type erreur \\
Paramètres~: &
\begin{tabular}[t]{@{\extracolsep{0pt}}ll}
    \textsl{v}~: & The value to display \\
  \end{tabular} \\
\end{tabular} \\[0.3cm]
\end{minipage}

\begin{minipage}{\linewidth}
\functitle{afficher\_action\_type}
\begin{lst-c++}
void afficher_action_type(action_type v)
\end{lst-c++}
\noindent
\begin{tabular}[t]{@{\extracolsep{0pt}}>{\bfseries}lp{10cm}}
Description~: & Affiche le contenu d'une valeur de type action\_type \\
Paramètres~: &
\begin{tabular}[t]{@{\extracolsep{0pt}}ll}
    \textsl{v}~: & The value to display \\
  \end{tabular} \\
\end{tabular} \\[0.3cm]
\end{minipage}

\begin{minipage}{\linewidth}
\functitle{afficher\_caracteristique}
\begin{lst-c++}
void afficher_caracteristique(caracteristique v)
\end{lst-c++}
\noindent
\begin{tabular}[t]{@{\extracolsep{0pt}}>{\bfseries}lp{10cm}}
Description~: & Affiche le contenu d'une valeur de type caracteristique \\
Paramètres~: &
\begin{tabular}[t]{@{\extracolsep{0pt}}ll}
    \textsl{v}~: & The value to display \\
  \end{tabular} \\
\end{tabular} \\[0.3cm]
\end{minipage}

\begin{minipage}{\linewidth}
\functitle{afficher\_debug\_chien}
\begin{lst-c++}
void afficher_debug_chien(debug_chien v)
\end{lst-c++}
\noindent
\begin{tabular}[t]{@{\extracolsep{0pt}}>{\bfseries}lp{10cm}}
Description~: & Affiche le contenu d'une valeur de type debug\_chien \\
Paramètres~: &
\begin{tabular}[t]{@{\extracolsep{0pt}}ll}
    \textsl{v}~: & The value to display \\
  \end{tabular} \\
\end{tabular} \\[0.3cm]
\end{minipage}

\begin{minipage}{\linewidth}
\functitle{afficher\_position}
\begin{lst-c++}
void afficher_position(position v)
\end{lst-c++}
\noindent
\begin{tabular}[t]{@{\extracolsep{0pt}}>{\bfseries}lp{10cm}}
Description~: & Affiche le contenu d'une valeur de type position \\
Paramètres~: &
\begin{tabular}[t]{@{\extracolsep{0pt}}ll}
    \textsl{v}~: & The value to display \\
  \end{tabular} \\
\end{tabular} \\[0.3cm]
\end{minipage}

\begin{minipage}{\linewidth}
\functitle{afficher\_plante}
\begin{lst-c++}
void afficher_plante(plante v)
\end{lst-c++}
\noindent
\begin{tabular}[t]{@{\extracolsep{0pt}}>{\bfseries}lp{10cm}}
Description~: & Affiche le contenu d'une valeur de type plante \\
Paramètres~: &
\begin{tabular}[t]{@{\extracolsep{0pt}}ll}
    \textsl{v}~: & The value to display \\
  \end{tabular} \\
\end{tabular} \\[0.3cm]
\end{minipage}

\begin{minipage}{\linewidth}
\functitle{afficher\_action\_hist}
\begin{lst-c++}
void afficher_action_hist(action_hist v)
\end{lst-c++}
\noindent
\begin{tabular}[t]{@{\extracolsep{0pt}}>{\bfseries}lp{10cm}}
Description~: & Affiche le contenu d'une valeur de type action\_hist \\
Paramètres~: &
\begin{tabular}[t]{@{\extracolsep{0pt}}ll}
    \textsl{v}~: & The value to display \\
  \end{tabular} \\
\end{tabular} \\[0.3cm]
\end{minipage}

